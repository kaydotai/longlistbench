The introduction should provide context for your research, explain the problem you're addressing, and outline the contributions of your work. Recent studies have explored various approaches to this problem~\cite{author2023title,author2023conference}. 

\subsection{Background and Motivation}
Describe the background of the problem and why it's important to solve. As noted by Smith~\cite{author2022book}, this area has seen significant development in recent years. Online resources~\cite{author2023web} provide additional context for understanding the scope of this challenge.

% Motivation:
% Growing popularity of AI automations. 
% So far the most most progress has been achieved on automation of document understanding. Need to find some reference for that, maybe some survey on AI automations.
% We at Kay.ai also discovered that document automation delivers the highest ROI and LLM models are the most advanced there (reference benchmark results for models maybe?), after trying Browser and Voice automations.
% The biggest ROI comes from working with large documents as an extraction pipeline developed ones can be used to process thousands of entries which would take days for humans (some more study on time working with docs), those large docs would have list entities. 

\subsection{Research Questions}
State your main research questions or hypotheses.

% There is no good benchmark that focuses completely on long list extraction. Main focus is Key Value extraction. (VRDU?)

\subsection{Contributions}
Clearly list the main contributions of this work:
\begin{itemize}
    \item First contribution
    \item Second contribution
    \item Third contribution
\end{itemize}

% Comprehensive dataset for long list eval
% Standardirized set of tools for eval

\subsection{Paper Organization}
The rest of this paper is organized as follows: Section~\ref{sec:conclusion} concludes the paper and discusses future work.

% weird, maybe delete it, need to compare to some paper. 
% Use CORD as sample cause it was accepted to NeurIPS