\section*{Appendix A: Evaluation Schemas}
\addcontentsline{toc}{section}{Appendix A: Evaluation Schemas}
\label{sec:appendix-schemas}

A recurring challenge with existing document extraction benchmarks is incomplete or ambiguous schema documentation. Without clear specifications, it can be difficult to understand expected field formats, handling of optional values, or normalization rules. Researchers attempting to reproduce results must often reverse-engineer these details from examples or evaluation scripts, leading to inconsistent implementations and incomparable metrics.

To ensure full reproducibility, we provide complete schemas with explicit type annotations, default values, and field descriptions. All schemas are implemented as Pydantic models, enabling automatic JSON Schema generation and runtime validation. The evaluation script validates model outputs against these schemas before scoring, ensuring that format errors are caught early rather than silently degrading metrics.

\subsection*{A.1 Financial Breakdown Schema}

Each incident contains four financial breakdown objects (\texttt{bi}, \texttt{pd}, \texttt{lae}, \texttt{ded}) with the following structure:

\begin{lstlisting}[language=Python, caption={FinancialBreakdown schema}]
from __future__ import annotations

from pydantic import BaseModel, Field


class FinancialBreakdown(BaseModel):
    reserve: float = Field(default=0.0, description="Amount reserved for potential payout")
    paid: float = Field(default=0.0, description="Amount already paid")
    recovered: float = Field(default=0.0, description="Amount recovered (e.g., deductible)")
    total_incurred: float = Field(default=0.0, description="Reserve + Paid - Recovered")
\end{lstlisting}

\subsection*{A.2 Loss Run Incident Schema}

The primary entity schema representing a single insurance claim incident:

\begin{lstlisting}[language=Python, caption={LossRunIncident schema}]
from typing import Optional

from pydantic import BaseModel, Field


class LossRunIncident(BaseModel):
    incident_number: str = Field(description="Incident number (e.g., #12345)")
    reference_number: str = Field(description="Reference ID (e.g., L240123)")
    company_name: str = Field(description="Trucking company name")
    division: str = Field(default="General", description="Company division")
    coverage_type: str = Field(description="Coverage type (Liability, Physical Damage, Inland Marine, Cargo)")
    status: str = Field(description="Open or Closed")
    policy_number: str = Field(description="Policy identifier")
    policy_state: str = Field(description="Policy state abbreviation")
    cause_code: Optional[str] = Field(default=None, description="Internal cause code")
    description: str = Field(description="Detailed incident description")
    handler: str = Field(default="Claims Adjuster", description="Claims handler")
    unit_number: Optional[str] = Field(default=None, description="Vehicle/truck unit ID")
    date_of_loss: str = Field(description="Date incident occurred")
    loss_state: str = Field(description="State where loss occurred")
    date_reported: str = Field(description="Date reported to insurance")
    agency: Optional[str] = Field(default=None, description="Insurance agency name")
    insured: str = Field(description="Insured party name")
    claimants: list[str] = Field(default_factory=list, description="List of claimants")
    driver_name: Optional[str] = Field(default=None, description="Driver name at time of incident")

    bi: FinancialBreakdown = Field(default_factory=FinancialBreakdown, description="Bodily Injury")
    pd: FinancialBreakdown = Field(default_factory=FinancialBreakdown, description="Property Damage")
    lae: FinancialBreakdown = Field(default_factory=FinancialBreakdown, description="Loss Adjustment Expense")
    ded: FinancialBreakdown = Field(default_factory=FinancialBreakdown, description="Deductible")

    adjuster_notes: Optional[str] = Field(default=None, description="Additional adjuster notes")
\end{lstlisting}

\subsection*{A.3 Extraction Output Schema}

Models are expected to return a JSON object matching the following structure:

\begin{lstlisting}[language=Python, caption={LossRunExtraction schema}]
class LossRunExtraction(BaseModel):
    incidents: list[LossRunIncident]
\end{lstlisting}

\subsection*{A.4 Field Scoring Rules}

During evaluation, fields are normalized and compared as follows:

\begin{itemize}
    \item \textbf{String fields}: Trimmed of whitespace. Optional string fields treat empty strings as \texttt{null}.
    \item \textbf{Numeric fields}: Rounded to two decimal places. Negative zero is normalized to zero.
    \item \textbf{List fields}: Sorted alphabetically for comparison.
    \item \textbf{Financial breakdowns}: Each sub-field (\path{reserve}, \path{paid}, \path{recovered}, \path{total_incurred}) is scored independently.
\end{itemize}

The evaluation computes field-level precision, recall, and F1 by flattening each incident into (\path{incident_id}, \path{field_path}, \path{value}) tuples and comparing predicted tuples against ground truth.
