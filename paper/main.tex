\documentclass[12pt]{article}

% Essential packages
\usepackage[utf8]{inputenc}
\usepackage[T1]{fontenc}
\usepackage{lmodern}
\usepackage[english]{babel}

% Page layout - arXiv compliant (US letter, 1" margins)
\usepackage[letterpaper,margin=1in]{geometry}

% Spacing - single space for arXiv (comment out for double-space drafts)
% \usepackage{setspace}
% \doublespacing

% Math packages
\usepackage{amsmath}
\usepackage{amssymb}
\usepackage{amsthm}

% Graphics and figures
\usepackage{graphicx}
\usepackage{float}
\usepackage{caption}
\usepackage{subcaption}

% Tables
\usepackage{booktabs}
\usepackage{multirow}
\usepackage{array}

% Links and references (hyperref should be loaded late)
\usepackage{xurl}
\usepackage{hyperref}
\hypersetup{
    colorlinks=true,
    linkcolor=blue,
    citecolor=blue,
    urlcolor=blue,
    pdfauthor={Anton Fedoruk, Serhii Shchoholiev, Akhil Mehta},
    pdftitle={LongListBench: A Benchmark for Long-List Entity Extraction Under Layout and OCR Noise},
    pdfsubject={Benchmark for long-list entity extraction under layout and OCR noise},
    pdfkeywords={document understanding, information extraction, OCR, large language models, benchmark, long-list entity extraction, layout noise},
    bookmarksnumbered=true,
    bookmarksopen=true,
    pdfstartview=FitH,
    pdfpagemode=UseOutlines,
}
\usepackage{cleveref}

% Bibliography
\usepackage{csquotes}
\usepackage[style=ieee,sorting=none]{biblatex}
\addbibresource{references.bib}

% Code listings (if needed)
\usepackage{listings}
\usepackage{xcolor}

% Define colors for pseudo code
\definecolor{schematype}{RGB}{0, 112, 192}      % Blue for types
\definecolor{schemaprop}{RGB}{128, 64, 0}       % Brown for properties
\definecolor{schemacomment}{RGB}{96, 96, 96}    % Gray for comments
\definecolor{schemastring}{RGB}{163, 21, 21}    % Red for string values

% Define pseudo code language
\lstdefinelanguage{schema}{
    keywords={String, Float, List, FinancialBreakdown, LossRunIncident, LossRunExtraction},
    keywordstyle=\color{schematype}\bfseries,
    morecomment=[l]{//},
    commentstyle=\color{schemacomment}\itshape,
    morestring=[b]",
    stringstyle=\color{schemastring},
    sensitive=true,
}

\lstset{
    basicstyle=\ttfamily\small,
    breaklines=true,
    breakatwhitespace=false,
    postbreak=\mbox{\textcolor{gray}{$\hookrightarrow$}\space},
    columns=fullflexible,
    keepspaces=true,
    showstringspaces=false,
    tabsize=4,
    frame=single,
    framesep=2pt,
    framerule=0.2pt,
    rulecolor=\color{black!20},
    backgroundcolor=\color{white},
    keywordstyle=\bfseries\color{schematype},
    commentstyle=\itshape\color{schemacomment},
    stringstyle=\color{schemastring},
    numbers=none,
    numberstyle=\tiny\color{gray},
    numbersep=10pt,
    xleftmargin=0.6em,
    framexleftmargin=0.6em,
    aboveskip=0.8\baselineskip,
    belowskip=0.8\baselineskip,
    language=schema,
}

% Custom theorem environments
\newtheorem{theorem}{Theorem}[section]
\newtheorem{lemma}[theorem]{Lemma}
\newtheorem{proposition}[theorem]{Proposition}
\newtheorem{corollary}[theorem]{Corollary}
\theoremstyle{definition}
\newtheorem{definition}[theorem]{Definition}
\newtheorem{example}[theorem]{Example}
\theoremstyle{remark}
\newtheorem{remark}[theorem]{Remark}

% Document metadata
\title{LongListBench: A Benchmark for Long-List Entity Extraction Under Layout and OCR Noise}
\author{
    Anton Fedoruk\textsuperscript{1}\thanks{\texttt{anton@kay.ai}}
    \hspace{2em}
    Serhii Shchoholiev\textsuperscript{1}\thanks{\texttt{serhii@kay.ai}}
    \hspace{2em}
    Akhil Mehta\textsuperscript{1}\thanks{\texttt{akhil@kay.ai}}
    \\[1em]
    \normalsize\textsuperscript{1}Kay.ai, Brooklyn, NY, USA
}
\date{\today}

\begin{document}

\maketitle

\begin{abstract}
Existing datasets for evaluating document extraction using large language models (LLMs) predominantly focus on key-value pair extraction and fail to address the complexities of long list entity extraction. However, real-world business documents such as invoices, insurance claims, purchase orders, and financial statements commonly organize data in table-like structures with repetitive entities of the same type---a critical gap in current benchmarking efforts. We introduce a comprehensive dataset specifically designed to evaluate long list extraction performance, incorporating common challenges observed in production environments, including true duplicate entities, multi-row entries, and various structural inconsistencies. Our dataset construction methodology leverages a corpus of real-world insurance claims gathered through Kay.ai operations, combined with systematically identified problematic patterns in table-like documents. We employ LLMs to generate realistic document layouts, which are subsequently rendered into PDFs and processed through optical character recognition (OCR) to simulate authentic extraction scenarios complete with real-world noise. Additionally, we provide standardized evaluation scripts to facilitate reproducible assessments. We benchmark flagship models from OpenAI, Anthropic, and Google using practical, easily implementable techniques including zero-shot prompting and accumulative generation. Our work addresses a significant gap in document understanding evaluation and provides the research community with essential tools for advancing long list extraction capabilities.

% TODO:
% Make examples more robust: invoices, purchase orders, and financial statements are pretty much the same
% Add note about using Redacto for evals (Can switch to extend instead, Achyut mentioned he talked to their CEO before, might be a better connection for publicity)
\end{abstract}

\section{Introduction}
\label{sec:introduction}
Long-list entity extraction---recovering dozens to hundreds of repeated records from semi-structured documents---is a core requirement for document automation in domains such as insurance, finance, and procurement. While recent advances in document understanding models (e.g., layout-aware pretraining~\cite{xu2020layoutlm} and OCR-free approaches~\cite{kim2022donut}) and general-purpose LLMs have improved extraction quality, robust evaluation of long-list scenarios remains limited.

Many established benchmarks focus on key-value style extraction or relatively short, form-like documents (e.g., FUNSD~\cite{jaume2019funsd}) or narrow document types such as receipts (SROIE~\cite{huang2021sroie}). More recent datasets such as DocILE~\cite{simsa2023docile} include business documents and line items, but long lists in the wild often exhibit additional failure modes: repeated entities, page breaks, multi-column reading order, irrelevant tables, and table constructs such as merged cells. VRDU~\cite{wang2023vrdu} highlights that hierarchical and long-list fields remain challenging for LLM-based extraction.

We introduce Lost-and-Found Entities, a benchmark designed to stress-test long-list extraction on loss run documents under systematically injected document phenomena and OCR noise.

\subsection{Background and Motivation}
In production workflows, a single loss run PDF can contain many incidents and associated financial breakdowns. Systems must extract a complete list of claim records (not merely a few key-value pairs) while handling complex layouts and noisy OCR. We aim to support research on extraction methods that remain reliable as list length grows and as layout artifacts accumulate.

% Motivation:
% Growing popularity of AI automations. 
% So far the most most progress has been achieved on automation of document understanding. Need to find some reference for that, maybe some survey on AI automations.
% We at Kay.ai also discovered that document automation delivers the highest ROI and LLM models are the most advanced there (reference benchmark results for models maybe?), after trying Browser and Voice automations.
% The biggest ROI comes from working with large documents as an extraction pipeline developed ones can be used to process thousands of entries which would take days for humans (some more study on time working with docs), those large docs would have list entities. 

\subsection{Research Questions}
This work is organized around three practical questions:
\begin{itemize}
    \item How do common long-list document phenomena (page breaks, duplicates, multi-row cells, multi-column layouts, irrelevant tables, merged cells) affect extraction quality?
    \item To what extent are end-to-end failures attributable to OCR transcription versus downstream extraction?
    \item How do strong off-the-shelf LLMs perform under a simple, reproducible zero-shot protocol?
\end{itemize}

% There is no good benchmark that focuses completely on long list extraction. Main focus is Key Value extraction. (VRDU?)

\subsection{Contributions}
We make the following contributions:
\begin{itemize}
    \item A reproducible benchmark generation pipeline that produces paired ground truth JSON, rendered PDFs, and OCR transcripts.
    \item A dataset of 80 documents (40 detailed, 40 table) containing 6{,}828 incident rows across four difficulty tiers, with an extreme tier reaching 500 incidents per document.
    \item A taxonomy of seven injected problem types and evaluation scripts for incident-level scoring and OCR identifier coverage.
    \item Baseline results for GPT-4o and Claude Sonnet 4 under a shared prompt, highlighting remaining gaps in long-list extraction.
\end{itemize}

% Comprehensive dataset for long list eval
% Standardirized set of tools for eval

\subsection{Paper Organization}
Section~\ref{sec:related-work} reviews relevant datasets and models. Section~\ref{sec:methodology} describes benchmark construction and the problem taxonomy. Section~\ref{sec:evaluation} presents the evaluation protocol. Section~\ref{sec:results} reports baseline results. Section~\ref{sec:limitations} discusses limitations and future directions, and Section~\ref{sec:conclusion} concludes.

% weird, maybe delete it, need to compare to some paper. 
% Use CORD as sample cause it was accepted to NeurIPS

\section{Related Work}
\label{sec:related-work}
% =============================================================================
% RELATED WORK - Papers to include and why
% =============================================================================

% -----------------------------------------------------------------------------
% 1. DOCUMENT UNDERSTANDING BENCHMARKS
% -----------------------------------------------------------------------------

% VRDU (Wang et al., 2023) - KDD 2023
% WHY: PRIMARY MOTIVATION - explicitly highlights that "models struggle with hierarchical
% fields such as line-items in an invoice." Contains Ad-buy Forms with hierarchical entity
% annotations - the only benchmark addressing repeated/nested fields extraction.
% URL: https://arxiv.org/abs/2211.15421
% Quote: "Extracting hierarchical or repeated entities is really challenging"
% Quote:  models struggle with hierarchical fields such as line-items in an invoice.



% TO BE REVIEWED

% DocVQA (Mathew et al., 2021) - WACV 2021
% WHY: Foundational benchmark for document visual question answering.
% Standard evaluation dataset, but focuses on QA rather than structured extraction.
% URL: https://www.docvqa.org/

% FUNSD (Jaume et al., 2019)
% WHY: Form understanding benchmark with 199 real-world scanned documents.
% Widely used but focuses on key-value pair extraction, not long lists.
% Used to show gap: existing benchmarks don't address our problem.

% CORD (Park et al., 2019) - GOT PUBLISHED AT NEURIPS 2019 - big deal
% WHY: Receipt understanding benchmark. Contains some repeated items but
% limited template diversity and simpler structure than real business documents.

% SROIE (Huang et al., 2019)
% WHY: Receipt information extraction. Similar to CORD - useful baseline
% but doesn't capture complexity of insurance claims or invoices.

% -----------------------------------------------------------------------------
% 2. LAYOUT-AWARE DOCUMENT MODELS (Pre-LLM Era)
% -----------------------------------------------------------------------------

% LayoutLM (Xu et al., 2020) - KDD 2020
% WHY: Pioneering work on multimodal pre-training combining text, layout, and image.
% Established foundation for document understanding but limited by encoder-only architecture.

% LayoutLMv2 (Xu et al., 2021) - ACL 2021
% WHY: Introduced spatial-aware self-attention. SOTA on FUNSD, CORD, SROIE, DocVQA.
% Important baseline showing pre-LLM capabilities.

% LayoutLMv3 (Huang et al., 2022) - ACM MM 2022
% WHY: Unified pre-training for text, layout, and image. F1=0.9029 on FUNSD.
% Best encoder-based model before LLM era.

% GraphLayoutLM (Li et al., 2023) - ACM MM 2023
% WHY: Uses graph structure modeling to capture layout relationships between text nodes.
% Shows importance of structural information beyond positional embeddings.

% -----------------------------------------------------------------------------
% 3. OCR-FREE DOCUMENT UNDERSTANDING
% -----------------------------------------------------------------------------

% Donut (Kim et al., 2022) - ECCV 2022
% WHY: OCR-free end-to-end transformer. Shows alternative to OCR pipeline.
% Important comparison: OCR-based vs OCR-free approaches.
% Limitations: struggles with complex layouts and long documents.
% URL: https://arxiv.org/abs/2111.15664

% -----------------------------------------------------------------------------
% 4. LLM-BASED DOCUMENT EXTRACTION (Most Relevant)
% -----------------------------------------------------------------------------

% LMDX (Perot et al., 2024) - ACL Findings 2024
% WHY: HIGHLY RELEVANT - Google's methodology to adapt LLMs for document extraction.
% Handles singular, repeated, AND hierarchical entities with localization.
% Sets SOTA on VRDU and CORD. Uses chunking for long documents.
% Key innovation: decoding algorithm that discards hallucinations.
% URL: https://arxiv.org/abs/2309.10952

% LayoutLLM (Luo et al., 2024) - CVPR 2024
% WHY: Layout instruction tuning with LLMs. Shows LLMs benefit from layout awareness.
% Achieved 70.82% on DocVQA, 70.96% on FUNSD in zero-shot setting.
% URL: https://openaccess.thecvf.com/content/CVPR2024/papers/Luo_LayoutLLM...

% DocLayLLM (2024)
% WHY: Efficient multimodal extension of LLMs for text-rich document understanding.
% Recent work showing continued advancement in LLM-based approaches.

% ARIAL (2024)
% WHY: Agentic framework for DocVQA with answer localization. SOTA results:
% 88.7 ANLS on DocVQA, 90.0 on FUNSD, 85.5 on CORD, 93.1 on SROIE.
% Shows modular approach with specialized tools.

% -----------------------------------------------------------------------------
% 5. TABLE EXTRACTION
% -----------------------------------------------------------------------------

% Table Transformer / PubTables-1M (Smock et al., 2022) - ICDAR 2023
% WHY: Deep learning for table extraction. GriTS evaluation metric.
% Relevant because tables are structured repeated entities.
% URL: https://github.com/microsoft/table-transformer

% TC-OCR (Anand et al., 2023) - MMIR Workshop 2023
% WHY: End-to-end table recognition pipeline. 25% improvement over Table Transformer.
% Shows table extraction challenges remain unsolved.

% TableFormer (Nassar et al., 2022)
% WHY: Table structure understanding with transformers.
% Relevant for comparison with our structured extraction approach.

% -----------------------------------------------------------------------------
% 6. MULTIMODAL VISION-LANGUAGE MODELS
% -----------------------------------------------------------------------------

% GPT-4V (OpenAI, 2023)
% WHY: We benchmark this. Multimodal capabilities for document understanding.
% Known limitations: hallucinations, errors in complex diagrams.
% URL: https://cdn.openai.com/papers/GPTV_System_Card.pdf

% Claude 3.5 Sonnet (Anthropic, 2024)
% WHY: We benchmark this. Best for data extraction per industry reports.
% Native PDF handling capability.

% Gemini Pro (Google, 2024)
% WHY: We benchmark this. Comparison point for LMDX results.

% MiniGPT-4, LLaVA (2023)
% WHY: Shows emergence of vision-language models. Context for multimodal approaches.

% -----------------------------------------------------------------------------
% 7. SURVEYS (Good for positioning our work)
% -----------------------------------------------------------------------------

% "Deep Learning for Visually Rich Documents" (2024) - IJDAR
% WHY: Comprehensive survey covering 100+ papers. Good for establishing context.
% Covers challenges: text recognition, layout analysis, information fusion.
% URL: https://link.springer.com/article/10.1007/s10032-024-00493-8

% "Survey of Form Understanding in Scanned Documents" (2024) - AI Review
% WHY: Analyzes 15 SOTA models and 10 benchmarks. Transformer models improved
% performance by 25% over traditional methods.
% URL: https://link.springer.com/article/10.1007/s10462-024-11000-0

% -----------------------------------------------------------------------------
% 8. STRUCTURED OUTPUT FROM LLMs
% -----------------------------------------------------------------------------

% Instructor Library / Schema Design Impact
% WHY: Shows field naming impacts performance drastically (4.5% -> 95%).
% Chain-of-thought boosts performance by 60% on GSM8k.
% Relevant for our prompting strategies.

% StructEval (2024)
% WHY: Benchmark for LLMs generating structural outputs (JSON, XML, etc.)
% Directly relevant to our structured extraction task.
% URL: https://arxiv.org/html/2505.20139v1

% -----------------------------------------------------------------------------
% KEY GAPS OUR PAPER ADDRESSES (based on related work):
% -----------------------------------------------------------------------------
% 1. VRDU is the only benchmark with hierarchical entities, but limited scale/diversity
% 2. No benchmark specifically designed for LONG LIST extraction (10+ entities)
% 3. Existing benchmarks don't include true duplicates or multi-row entries
% 4. No standardized evaluation for accumulative/iterative generation approaches
% 5. Insurance claims domain is underrepresented in academic benchmarks
% 6. OCR noise + complex layouts + long lists = unexplored combination
% -----------------------------------------------------------------------------

Research on information extraction (IE) from visually rich documents has produced a broad ecosystem of datasets and models. However, much of the public evaluation landscape emphasizes either short documents (e.g., forms) or key-value extraction, leaving long-list entity extraction underexplored.

\subsection{Document IE benchmarks}
Early and widely used benchmarks such as FUNSD~\cite{jaume2019funsd} focus on form understanding in noisy scans. Receipt datasets and challenges such as SROIE~\cite{huang2021sroie} emphasize OCR and key fields in narrow document types. These benchmarks are valuable, but typically contain relatively short documents and do not directly stress long lists of repeated entities.

DocILE~\cite{simsa2023docile} broadens the scope to business documents and includes line-item recognition, which is closer in spirit to long-list extraction. VRDU~\cite{wang2023vrdu} further argues that hierarchical and repeated fields (e.g., invoice line items) remain difficult for LLM-based extraction. Our benchmark complements these efforts by focusing on list length, repeated entity boundaries, and a targeted taxonomy of long-list failure modes.

\subsection{Document understanding models}
Layout-aware pretraining approaches such as LayoutLM~\cite{xu2020layoutlm} jointly model textual content and 2D document structure, yielding strong performance on a range of document understanding tasks. In parallel, OCR-free approaches such as Donut~\cite{kim2022donut} avoid explicit OCR by directly generating structured outputs from document images, mitigating OCR error propagation at the cost of specialized training.

In contrast, our work is model-agnostic: we provide paired PDF, OCR transcript, and ground truth, enabling evaluation of OCR-based pipelines, OCR-free models, and LLM-based extraction. Our primary goal is to support reproducible measurement of long-list extraction robustness under realistic layout artifacts.


\section{Benchmark Construction}
\label{sec:methodology}
% About how we syntesize the dataset
% List all problems we identified that can occur in long list data
% Include visual for generation pipeline

\section{Evaluation}
\label{sec:evaluation}
% explain how models will be evaluated against the dataset
% zero-shot
% accumulative generation (continue)
% Test Redacto - get their attention 

% tell about eval tools created for this (they will need to be added to this repo and open sourced) 

% results of evals go into next chapter

\section{Results}
\label{sec:results}
We summarize results for (i) OCR fidelity and (ii) baseline extraction performance. OCR coverage and LLM baseline numbers are produced by the released scripts in the repository.

\subsection{OCR identifier coverage}
Using \texttt{benchmarks/validate\_ocr\_vs\_golden.py} we measure how often key identifiers from the ground truth appear verbatim in the OCR transcript. Across the full dataset (80 OCR transcripts), incident numbers and reference numbers exhibit 100\% coverage (mean and minimum). These results indicate that, for primary identifiers, our OCR step rarely drops information and that most downstream failures are attributable to extraction rather than transcription.

\begin{table}[t]
\centering
\caption{OCR identifier coverage on the full dataset (80 documents).}
\label{tab:ocr-coverage}
\begin{tabular}{lrr}
\toprule
Identifier & Mean coverage & Min coverage \\
\midrule
Incident number & 100.0\% & 100.0\% \\
Reference number & 100.0\% & 100.0\% \\
\bottomrule
\end{tabular}
\end{table}

\subsection{Zero-shot LLM extraction baseline}
We evaluate three LLMs using the shared prompt and evaluation harness in \texttt{benchmarks/evaluate\_models.py}. The current report (\texttt{benchmarks/results/evaluation\_report.*}) covers three detailed-format samples (one hard and two extreme), and uses schema-conformant, field-level scoring. Over this subset, GPT-4o achieves 89.1\% average F1 (87.8\% recall, 90.6\% precision), GPT-5.2 achieves 82.9\% average F1 (81.5\% recall, 84.4\% precision), and Gemini 2.0 Flash achieves 86.5\% average F1 (85.0\% recall, 88.1\% precision).

\begin{table}[t]
\centering
\caption{Zero-shot LLM baseline results on three detailed-format samples under schema-conformant, field-level scoring (\texttt{benchmarks/results/evaluation\_report.json}).}
\label{tab:llm-baselines}
\begin{tabular}{lrrrr}
\toprule
Model & Samples & Avg Recall & Avg Precision & Avg F1 \\
\midrule
GPT-4o & 3 & 87.8\% & 90.6\% & 89.1\% \\
GPT-5.2 & 3 & 81.5\% & 84.4\% & 82.9\% \\
Gemini 2.0 Flash & 3 & 85.0\% & 88.1\% & 86.5\% \\
\bottomrule
\end{tabular}
\end{table}

Qualitatively, errors often manifest as local field-level deviations (e.g., missing optional strings, numeric drift in financial breakdowns, or small identifier formatting mistakes) spread across an otherwise correct long list.

These findings suggest that recovering identifiers is largely deterministic under our OCR pipeline, while the main open challenge for long-list extraction is robustly segmenting and populating full per-incident records under layout disruptions (page breaks, multi-column order, irrelevant tables, merged cells) and scale (hundreds of incidents).

\section{Limitations and Future Directions}
\label{sec:limitations}
% Limitations of our dataset? 
% I am actually not sure, maybe the fact that its syntetic?

\section{Conclusion}
\label{sec:conclusion}
Summarize your work, restate the main contributions, and suggest directions for future research.

% Main contribution - a comprehensive dataset to evaluate pipeline / model ability to extract long list entities
% How can long list extraction be even more complex? - for future research

\subsection{Summary}
Briefly summarize the key points of your paper.

\subsection{Future Work}
Suggest directions for future research.



% \section*{Acknowledgments}
% We thank colleagues for helpful feedback on early drafts.
%


\printbibliography

\newpage
\section*{Appendix A: Evaluation Schemas}
\addcontentsline{toc}{section}{Appendix A: Evaluation Schemas}
\label{sec:appendix-schemas}

A recurring challenge with existing document extraction benchmarks is incomplete or ambiguous schema documentation. Without clear specifications, it can be difficult to understand expected field formats, handling of optional values, or normalization rules. Researchers attempting to reproduce results must often reverse-engineer these details from examples or evaluation scripts, leading to inconsistent implementations and incomparable metrics.

To ensure full reproducibility, we provide complete schemas with explicit type annotations, default values, and field descriptions. All schemas are implemented as Pydantic models, enabling automatic JSON Schema generation and runtime validation. The evaluation script validates model outputs against these schemas before scoring, ensuring that format errors are caught early rather than silently degrading metrics.

\subsection*{A.1 Financial Breakdown Schema}

Each incident contains four financial breakdown objects (\texttt{bi}, \texttt{pd}, \texttt{lae}, \texttt{ded}) with the following structure:

\begin{lstlisting}[language=Python, caption={FinancialBreakdown schema}]
class FinancialBreakdown(BaseModel):
    reserve: float = 0.0
        # Amount reserved for potential payout
    paid: float = 0.0
        # Amount already paid
    recovered: float = 0.0
        # Amount recovered (e.g., deductible)
    total_incurred: float = 0.0
        # Reserve + Paid - Recovered
\end{lstlisting}

\subsection*{A.2 Loss Run Incident Schema}

The primary entity schema representing a single insurance claim incident:

\begin{lstlisting}[language=Python, caption={LossRunIncident schema}]
class LossRunIncident(BaseModel):
    # Identifiers
    incident_number: str
        # Incident number (e.g., #12345)
    reference_number: str
        # Reference ID (e.g., L240123)

    # Company information
    company_name: str
        # Trucking company name
    division: str = "General"
        # Company division
    insured: str
        # Insured party name
    agency: Optional[str] = None
        # Insurance agency name

    # Policy details
    policy_number: str
        # Policy identifier
    policy_state: str
        # Policy state abbreviation
    coverage_type: str
        # Coverage type (Liability, Physical Damage,
        # Inland Marine, Cargo)
    status: str
        # Open or Closed

    # Incident details
    description: str
        # Detailed incident description
    cause_code: Optional[str] = None
        # Internal cause code
    date_of_loss: str
        # Date incident occurred
    date_reported: str
        # Date reported to insurance
    loss_state: str
        # State where loss occurred

    # Personnel
    handler: str = "Claims Adjuster"
        # Claims handler
    driver_name: Optional[str] = None
        # Driver name at time of incident
    claimants: list[str] = []
        # List of claimants

    # Vehicle
    unit_number: Optional[str] = None
        # Vehicle/truck unit ID

    # Financial breakdowns
    bi: FinancialBreakdown
        # Bodily Injury
    pd: FinancialBreakdown
        # Property Damage
    lae: FinancialBreakdown
        # Loss Adjustment Expense
    ded: FinancialBreakdown
        # Deductible

    # Notes
    adjuster_notes: Optional[str] = None
        # Additional adjuster notes
\end{lstlisting}

\subsection*{A.3 Extraction Output Schema}

Models are expected to return a JSON object matching the following structure:

\begin{lstlisting}[language=Python, caption={LossRunExtraction schema}]
class LossRunExtraction(BaseModel):
    incidents: list[LossRunIncident]
\end{lstlisting}

\subsection*{A.4 Field Scoring Rules}

During evaluation, fields are normalized and compared as follows:

\begin{itemize}
    \item \textbf{String fields}: Trimmed of whitespace. Optional string fields treat empty strings as \texttt{null}.
    \item \textbf{Numeric fields}: Rounded to two decimal places. Negative zero is normalized to zero.
    \item \textbf{List fields}: Sorted alphabetically for comparison.
    \item \textbf{Financial breakdowns}: Each sub-field (\texttt{reserve}, \texttt{paid}, \texttt{recovered}, \texttt{total\_incurred}) is scored independently.
\end{itemize}

The evaluation computes field-level precision, recall, and F1 by flattening each incident into (incident\_id, field\_path, value) tuples and comparing predicted tuples against ground truth.


\end{document}

