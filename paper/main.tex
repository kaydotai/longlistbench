\documentclass[12pt,a4paper]{article}

% Essential packages
\usepackage[utf8]{inputenc}
\usepackage[T1]{fontenc}
\usepackage{lmodern}
\usepackage[english]{babel}

% Page layout
\usepackage[margin=1in]{geometry}
\usepackage{setspace}
\onehalfspacing

% Math packages
\usepackage{amsmath}
\usepackage{amssymb}
\usepackage{amsthm}

% Graphics and figures
\usepackage{graphicx}
\usepackage{float}
\usepackage{caption}
\usepackage{subcaption}

% Tables
\usepackage{booktabs}
\usepackage{multirow}
\usepackage{array}

% Links and references
\usepackage{hyperref}
\hypersetup{
    colorlinks=true,
    linkcolor=blue,
    citecolor=blue,
    urlcolor=blue,
}
\usepackage{cleveref}

% Bibliography
\usepackage[style=ieee,sorting=none]{biblatex}
\addbibresource{references.bib}

% Code listings (if needed)
\usepackage{listings}
\usepackage{xcolor}
\lstset{
    basicstyle=\ttfamily\small,
    breaklines=true,
    frame=single,
    numbers=left,
    numberstyle=\tiny\color{gray},
}

% Custom theorem environments
\newtheorem{theorem}{Theorem}[section]
\newtheorem{lemma}[theorem]{Lemma}
\newtheorem{proposition}[theorem]{Proposition}
\newtheorem{corollary}[theorem]{Corollary}
\theoremstyle{definition}
\newtheorem{definition}[theorem]{Definition}
\newtheorem{example}[theorem]{Example}
\theoremstyle{remark}
\newtheorem{remark}[theorem]{Remark}

% Document metadata
\title{Your Research Paper Title}
\author{
    Author Name\textsuperscript{1} \and
    Co-Author Name\textsuperscript{2}
}
\date{\today}

\begin{document}

\maketitle

\begin{abstract}
This is the abstract of your research paper. It should provide a brief summary of the research problem, methodology, key findings, and contributions. Typically, an abstract is between 150-250 words and provides readers with a quick overview of the paper's content without requiring them to read the entire document.
\end{abstract}

\section{Introduction}
\label{sec:introduction}

The introduction should provide context for your research, explain the problem you're addressing, and outline the contributions of your work. 

\subsection{Background and Motivation}
Describe the background of the problem and why it's important to solve.

\subsection{Research Questions}
State your main research questions or hypotheses.

\subsection{Contributions}
Clearly list the main contributions of this work:
\begin{itemize}
    \item First contribution
    \item Second contribution
    \item Third contribution
\end{itemize}

\subsection{Paper Organization}
The rest of this paper is organized as follows: Section~\ref{sec:related-work} reviews related work, Section~\ref{sec:methodology} describes our methodology, Section~\ref{sec:results} presents the results, Section~\ref{sec:discussion} discusses the findings, and Section~\ref{sec:conclusion} concludes the paper.

\section{Related Work}
\label{sec:related-work}

Review and discuss prior research relevant to your work. Compare and contrast different approaches, and explain how your work differs from or builds upon existing research.

\section{Methodology}
\label{sec:methodology}

Describe your research methodology in detail. This section should be comprehensive enough that other researchers could replicate your work.

\subsection{Problem Formulation}
Formally define the problem you're addressing.

\subsection{Proposed Approach}
Describe your proposed solution or approach.

\subsection{Implementation Details}
Provide specific implementation details, algorithms, or procedures.

\section{Experimental Setup}
\label{sec:experimental-setup}

Describe your experimental design, datasets, evaluation metrics, and implementation details.

\subsection{Datasets}
Describe the datasets used in your experiments.

\subsection{Evaluation Metrics}
Define the metrics used to evaluate your approach.

\subsection{Baseline Methods}
Describe the baseline methods you're comparing against.

\section{Results}
\label{sec:results}

Present your experimental results. Use tables, figures, and graphs to illustrate your findings.

\subsection{Main Results}
Present your main experimental results.

\begin{table}[H]
\centering
\caption{Example results table}
\label{tab:results}
\begin{tabular}{@{}lcc@{}}
\toprule
Method & Metric 1 & Metric 2 \\ \midrule
Baseline & 0.85 & 0.72 \\
Proposed & \textbf{0.92} & \textbf{0.85} \\ \bottomrule
\end{tabular}
\end{table}

\subsection{Ablation Study}
Present ablation studies to understand the contribution of different components.

\section{Discussion}
\label{sec:discussion}

Discuss the implications of your results, their significance, and any limitations of your approach.

\subsection{Analysis}
Provide detailed analysis of your results.

\subsection{Limitations}
Discuss the limitations of your work.

\section{Conclusion}
\label{sec:conclusion}

Summarize your work, restate the main contributions, and suggest directions for future research.

\subsection{Summary}
Briefly summarize the key points of your paper.

\subsection{Future Work}
Suggest directions for future research.

\section*{Acknowledgments}
This work was supported by [funding source]. We thank [people/organizations] for their contributions.

\printbibliography

\end{document}

